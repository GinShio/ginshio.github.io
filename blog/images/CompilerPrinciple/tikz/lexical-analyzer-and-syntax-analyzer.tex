\documentclass[border=10pt,margin=5pt,tikz,dvisvgm,rgb,utf8]{standalone}
\usepackage{ctex,xeCJK}  % 中文环境
\setCJKmainfont[BoldFont=Source Han Sans SC]{Source Han Serif SC}
\usepackage{calc,fontawesome,forest,smartdiagram,xcolor}
\usetikzlibrary{animations,arrows,automata,graphs,matrix,positioning,shadows,shapes}

\begin{document}
\renewcommand{\baselinestretch}{0.4}

\begin{tikzpicture}
  \node[align=center, text width=3em](source){源程序};
  \node[align=center, draw=black, text width=3em, right=2em of source](lexical_analyzer){词法\\分析器};
  \node[align=center, draw=black, text width=3em, right=5.5em of lexical_analyzer](syntax_analyzer){语法\\分析器};
  \node[align=center, draw=black, text width=3em, below=3em of $(lexical_analyzer)!0.5!(syntax_analyzer)$](symbol_table){符号表};
  \node[text width=-3pt, right=1em of syntax_analyzer](end){};

  \draw[->](source) -- (lexical_analyzer);
  \draw (lexical_analyzer) edge[->,bend left=10] node[above=-2pt]{\tiny 词法单元} (syntax_analyzer);
  \draw (syntax_analyzer) edge[->,bend left=10] node[below=-2pt]{\tiny getNextToken} (lexical_analyzer);
  \draw[<->](symbol_table) -- (lexical_analyzer);
  \draw[<->](symbol_table) -- (syntax_analyzer);
  \draw[->](syntax_analyzer) -- (end);
\end{tikzpicture}

\end{document}
